\section{Schätzung der Irradianz} % (fold)
\label{sec:schätzung_der_irradianz}



	\subsection{Analyse von Materialien} % (fold)
	\label{sub:analyse_von_materialien}

		Wie bereits in Abschnitt \ref{sub:bsdf} erwähnt, ist es im Allgemeinen nicht möglich, die BSDF eines Materials in geschlossener Form anzugeben.
		Es gibt jedoch verschiedene Verfahren um einfache BSDF-Modelle zu konstruieren, die in der Simulation der Strahldichte Anwendung finden.
		In \cite[S.~507~f]{pbrt3} wird hierfür die Verwendung von Messdaten, phenomenologischen Beobachtungen, Simulationsergebnissen und physikalischen Gesetzen als Beispiel genannt.
		Durch die Kombination solcher Modelle ist man dann in der Lage auch diverse komplexe Materialien zu simulieren.
		\begin{theorem}[zusammengesetzte BSDF]
			Seien $\mu\in\ssp$, $(f_n)_{n\in\SN}$ eine Folge von BSDFs bezüglich $\mu$ und $(\alpha_n)_{n\in\SN}$ eine Folge von Werten in $[0,1]$, sodass die beiden folgenden Eigenschaften für $\sigma^2$-fast-alle $(\omega_\m{i},\omega_\m{o})\in\ssp\times\ssp$ gelten.
			\[
				\sum_{n\in\SN}\alpha_n f_n(\omega_\m{i},\omega_\m{o}) < \infty ,\qquad \sum_{n\in\SN}\alpha_n \leq 1
			\]
			Dann ist auch die Abbildung $f$ mit der folgenden Definition eine BSDF bezüglich $\mu$.
			% \[
			% 	\func{f}{\ssp\times\ssp}{[0,\infty)},\qquad f\define \sum_{n\in\SN}\alpha_n f_n
			% \]
			\[
				\func{f}{\ssp\times\ssp}{[0,\infty)},\quad f(\omega_\m{i},\omega_\m{o})\define
				\begin{cases}
					\sum_{n\in\SN}\alpha_n f_n(\omega_\m{i},\omega_\m{o}) &: \sum_{n\in\SN}\alpha_n f_n(\omega_\m{i},\omega_\m{o}) < \infty \\
					0 &: \m{sonst}
				\end{cases}
			\]
		\end{theorem}
		\begin{proof}
			Für die Wohldefiniertheit von $f$ betrachten wir dessen Definitions- und Wertebereich.
			Der Definitionsbereich von $f$ entspricht dem der $f_n$ für alle $n\in\SN$.
			Weiterhin folgt aus der Definition von $f$, dass $f < \infty$  gilt.
			Für $n\in\SN$ betrachtet man dann
			\[
				f_n \geq 0 \quad \implies \quad \alpha_n f_n \geq 0 \quad \implies \quad f = \sum_{n\in\SN}\alpha_n f_n \geq 0
			\]
			Insbesondere ist also $f(\omega_\m{i},\omega_\m{o})\in[0,\infty)$ für alle $\omega_\m{i},\omega_\m{o}\in\ssp$.
			Damit ist $f$ eine Abbildung der Form $\func{f}{\ssp\times\ssp}{[0,\infty)}$ und wohldefiniert.

			Kommen wir nun zur Integrierbarkeit.
			Für alle $n\in\SN$ sind die Abbildungen $f_n$ und infolgedessen auch $\alpha_nf_n$ integrierbar.
			Wir definieren die Folge $(g_n)_{n\in\SN}$ von Funktionen durch
			\[
				g_n\define \sum_{i=1}^n \alpha_i f_i
			\]
			Dann ist $g_n$ aufgrund der Linearität des Integrals integrierbar und es gilt $g_n\leq g_{n+1}$ für alle $n\in\SN$.
			Weiterhin erhält man durch die Anwendung der Definition die folgende Aussage für $\sigma^2$-fast-alle $(\omega_\m{i},\omega_\m{o})\in\ssp\times\ssp$.
			\[
				g_n(\omega_\m{i},\omega_\m{o}) \conv[n\conv\infty] f(\omega_\m{i},\omega_\m{o})
			\]
			Nach dem Satz über die Monotone Konvergenz \cite[S.~125]{measure-theory} ist damit $f$ eine integrierbare Funktion, für die das Folgende aufgrund der Linearität des Integrals gilt.
			% \[
			% 	\Integral{\ssp\times\ssp}{}{f}{\sigma} = \lim_{n\conv\infty} \Integral{\ssp\times\ssp}{}{g_n}{\sigma} = \lim_{n\conv\infty} \Integral{\ssp\times\ssp}{}{\sum_{i=1}^n \alpha_i f_i}{\sigma} =
			% \]
			\begin{alignat*}{3}
				\integral{\ssp\times\ssp}{}{f}{\sigma^2} &=&&\ \lim_{n\conv\infty} \integral{\ssp\times\ssp}{}{g_n}{\sigma^2} = \lim_{n\conv\infty} \integral{\ssp\times\ssp}{}{\sum_{i=1}^n \alpha_i f_i}{\sigma^2} \\
				&=&&\ \lim_{n\conv\infty} \sum_{i=1}^n \alpha_i \integral{\ssp\times\ssp}{}{f_i}{\sigma^2} = \sum_{n\in\SN} \alpha_n \integral{\ssp\times\ssp}{}{f_n}{\sigma^2}
			\end{alignat*}

			Die Helmholtz-Reziprozität von $f$ wird direkt durch die Verwendung der Helmholtz-Reziprozität der $f_n$,$n\in\SN$ klar.
			Für $\sigma^2$-fast-alle $(\omega_\m{i},\omega_\m{o})$ mit $\omega_\m{i}\in\ssp$, $\omega_\m{o}\in\shs{\nu}$ und $\nu\define\sign(\dotp{\mu}{\omega_\m{i}})\cdot\mu$ gilt demnach
			\[
				f(\omega_\m{i},\omega_\m{o}) = \sum_{n\in\SN}\alpha_nf_n(\omega_\m{i},\omega_\m{o}) = \sum_{n\in\SN}\alpha_nf_n(\omega_\m{o},\omega_\m{i}) = f(\omega_\m{o},\omega_\m{i})
			\]

			Für die Energieerhaltung erhalten wir eine entsprechende Aussage durch die Anwendung des Satzes von Fubini \cite[S.~175~f]{measure-theory}.
			Es ist damit $f(\omega_\m{i},\cdot)$ $\sigma$-integrierbar für $\sigma$-fast-alle $\omega_\m{i}\in\ssp$.
			Die Funktion $\abs{\dotp{\mu}{\cdot}}$ ist stetig und durch $1$ beschränkt.
			Mithin ist auch $f(\omega_\m{i},\cdot)\abs{\dotp{\mu}{\cdot}}$ messbar und es gilt für $\sigma$-fast-alle $\omega_\m{i}\in\ssp$
			\[
				f(\omega_\m{i},\cdot)\abs{\dotp{\mu}{\cdot}} \leq f(\omega_\m{i},\cdot) \quad \implies \quad \integral{\ssp}{}{f(\omega_\m{i},\cdot)\abs{\dotp{\mu}{\cdot}}}{\sigma} \leq \integral{\ssp}{}{f(\omega_\m{i},\cdot)}{\sigma} < \infty
			\]
			Analog zum Beweis der Integrierbarkeit von $f$ lässt sich dann mithilfe der Energieerhaltung der $f_n$,$n\in\SN$ die Energieerhaltung von $f$ formulieren.
			\[
				\integral{\ssp}{}{f(\omega_\m{i},\cdot)\abs{\dotp{\mu}{\cdot}}}{\sigma} = \sum_{n\in\SN}\alpha_n \integral{\ssp}{}{f_n(\omega_\m{i},\cdot)\abs{\dotp{\mu}{\cdot}}}{\sigma} \leq \sum_{n\in\SN}\alpha_n \leq 1
			\]
		\end{proof}

		Das Theorem beweist, dass man sogar in der Lage ist abzählbar viele BSDFs miteinander zu kombinieren, sofern man auf die Konvergenz der Koeffizienten achtet.
		In den meisten Fällen reicht es allerdings endlich viele BSDFs zu betrachten.

		\begin{corollary}

		\end{corollary}

	% subsection analyse_von_materialien (end)

% section schätzung_der_irradianz (end)