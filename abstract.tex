Der Gegenstand dieser Arbeit ist es, eine Datenstruktur zu entwickeln, die die Lichtverteilung eines Modells speichert, um unnötige Berechnugen, die bei der Simulation globaler Beleuchtungseffekte im Regelfall ausgeführt werden müssen, zu eliminieren.
Dies erlaubt, das Modell von beliebigen Punkten im Raum aus zu beobachten, ohne dessen gesamte Beleuchtung ständig neu berechnen zu müssen.
Dafür nehme ich eine genauere Betrachtung der Irradianzbestimmung vor und konstruiere einen Algorithmus, der die erhaltenen Werte auf der Oberfläche des Modells speichert.
Das Verfahren ermöglicht damit unter Verwendung zusätzlichen Speichers und einer gewissen Vorberechnungszeit die simultane Darstellung der Beleuchtungsverteilung ohne störende Rauscheffekte, wie sie für andere Verfahren typisch sind.