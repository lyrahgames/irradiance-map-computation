\section{Einleitung} % (fold)
\label{sec:introduktion}

	\begin{figure}[b]
		% \center
		\begin{subfigure}[t]{0.33\textwidth}
			\center
			\includegraphics[width=0.95\textwidth]{pic/intro-shaderball-pt-1024-2.png}
			\caption{\parbox[t]{0.5\textwidth}{Path Tracing}}
		\end{subfigure}
		\begin{subfigure}[t]{0.33\textwidth}
			\center
			\includegraphics[width=0.95\textwidth]{pic/intro-shaderball-lightmap-462-2.png}
			\caption{\parbox[t]{0.5\textwidth}{Irradiance Map \\ Größe: $\sim462\unit{MiB}$}}
			\label{subfig:intro-shaderball-lm-big}
		\end{subfigure}
		\begin{subfigure}[t]{0.33\textwidth}
			\center
			\includegraphics[width=0.95\textwidth]{pic/intro-shaderball-lightmap-2-2.png}
			\caption{\parbox[t]{0.5\textwidth}{Irradiance Map \\ Größe: $\sim2\unit{MiB}$}}
			\label{subfig:intro-shaderball-lm-small}
		\end{subfigure}
		\caption[\enquote{Shaderball}-Szene mit einfachen Irradiance Maps]{Die Bilder zeigen alle denselben Ausschnitt der \enquote{Shaderball}-Szene mit unterschiedlichen Shading-Verfahren. Die Lichtquellen bestehen aus einer Sonne und einer Umgebungsbeleuchtung.}
		\label{fig:intro-shaderball}
	\end{figure}

	In der 3D-Computergrafik ist für die Erzeugung von realistischen Bildern die Simulation globaler Beleuchtungseffekte notwendig \cite{pbrt3}.
	Diese Lichteffekte ergeben sich formal als Lösung der \enquote{Rendergleichung} \cite{kajiya-lte}.
	Seit der Einführung dieser Gleichung im Jahre 1986 wurden verschiedene Algorithmen entwickelt, welche diese für beliebige Szenen numerisch lösen.
	Herauskristallisiert hat sich vor allem das \enquote{Path Tracing} \cite{pbrt3,kajiya-lte}.
	Dieses Verfahren kann die reale Beleuchtung beliebig genau und erwartungstreu schätzen.
	Aus diesem Grund werden die durch Path Tracing generierten Bilder meist als Referenzbild für die Bilder anderer Algorithmen verwendet, um deren Qualität zu untersuchen.

	Um nun verschiedene Materialien von Objekten simulieren zu können, werden häufig verschiedene Arten von Lichtstreuung an Oberflächen betrachtet.
	Eine der wichtigsten Arten ist gerade die ideale diffuse Streuung, welche der Szene ein grundlegendes plastisches Aussehen gibt.
	Sie ist unabhängig von der Richtung des einfallenden Lichtstrahls und damit auch invariant unter Änderung des Beobachtungspunktes \cite{intro-radiometry}.
	Für Path Tracing bedeutet dies, dass für jede Änderung der Kamera die eigentlich konstante Lichtverteilung neu berechnet werden muss.
	Diese Evaluierung nimmt aber auch den größten Rechenaufwand in Anspruch, da für jeden dieser Punkte Licht aus dessen gesamten Hemisphere eingesammelt werden muss.
	Um also das Verfahren des Path Tracings zu optimieren, müssten die diffusen Lichtverhältnisse vorberechnet und auf der Oberfläche der Szene gespeichert werden \cite{irradiance-caching}.

	Besonders in der Computerspieleindustrie wird dieses Problem mithilfe von sogenannten \enquote{Lightmaps} gelöst, welche für viele Punkte der Szene deren \enquote{Irradianz} in einer Textur speichern \cite{tricks-game}.
	Während des Renderings werden diese Irradianzen dann zusammen mit der Farbe der Materialtextur ausgelesen, miteinander multipliziert und dargestellt.
	Die Generierung einer solchen Lightmap ist jedoch mit diversen Tücken verbunden, welche in vielen Fällen nur durch manuelle Optimierung beseitigt werden können.

	Aus diesem Grund führe ich im Laufe dieser Arbeit die sogenannten \enquote{Irradiance Maps} ein, die Irradianzen eines Punktes auf der Oberfläche der zugehörigen Szene speichern.
	Die benötigte Datenstruktur soll dabei speziell für die Verwendung von \enquote{Raytracing} und Path Tracing ausgelegt sein.
	Ich werde eine genauere Betrachtung der Irradianzbestimmung vornehmen, um so deren Prozess zu vereinfachen.
	Des Weiteren präsentiere ich einen Algorithmus zur adaptiven und automatischen Generierung einer solchen Irradiance Map.

	Abbildung \ref{fig:intro-shaderball} (Quelle aller Abbildungen: Markus Pawellek 2017) zeigt ein Beispiel, in welchem deutlich wird, dass Irradiance Maps in der Lage sind die diffusen Lichtverhältnisse sehr gut zu approximieren, jedoch bei schlechter Generierung viel Speicher benötigen (siehe \ref{subfig:intro-shaderball-lm-big}).
	Außerdem ist in \ref{subfig:intro-shaderball-lm-small} erkennbar, dass für einen Großteil der Szene eine vergleichsweise kleine Auflösung der Irradiance Map ausreicht.
	Es wird also auch darum gehen, diesen Sachverhalt bei der Konstruktion auszunutzen.

	% \bigskip

	% Kurze Einführung: Was ist das Problem und warum ist es schwierig zu lösen?
	% Shading von Szenen mithilfe von Path Tracing benötigt enorme Menge an Strahlen.
	% Viele dieser Strahlen sind jedoch unabhängig vom Beobachter und können damit auch vorberechnet werden.
	% Die Vorberechnung muss jedoch auf der gesamten Oberfläche des Modells erfolgen und besitzt einen großen Speicherbedarf und eine große Berechnungszeit.
	% Des Weiteren wurde das bisherige Verfahren der lightmap Generierung immer manuell ausgeführt, was noch größeren Aufwand bedeutet.
	% Das Problem wird hier teilweise durch einen adaptiven Generator verhindert.
	% Dieser generiert eine perfekt passende lightmap anhand von n Parametern für jede Szene.
	% Hier vielleicht ein Bild als Beispiel mit einem Vergleich von purem path tracing und ray tracing mit lightmap mit fps.

	% Warum ist es nötig dieses Problem zu lösen?
	% Betrachte den Bereich CAD und die Computerspieleindustrie.
	% In beiden befinden sich meistens Szenen, welche sich kaum verändern und global beleuchtet werden.
	% Eine Vorberechung würde das möglich machen.
	% Eine schnelle adaptive Generierung spart viel Zeit, Arbeitskraft und Resourcen (Speicher usw.).
	% Vielleicht auch ein Beispielbild aus dem Internet mit Quellenangabe.

	% In beiden Paragraphen müssen Quellen angegeben werden, welche beweisen, dass es dieses Problem gibt.

% section introduktion (end)